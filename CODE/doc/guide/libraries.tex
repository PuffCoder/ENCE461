\chapter{Libraries}

\file{mmculib} is a library of C drivers, mostly for performing high-level I/O.
It is written to be microcontroller neutral.

\file{mat91lib} is a library of C drivers specifically for interfacing
with the peripherals of Atmel AT91 microcontrollers such as the Atmel
SAM4S. It provides the hardware abstraction layer.

%\textbf{Please do not edit the files in the mat91lib and mmculib}
%directories since this can lead to merge problems in the future.
If you find a bug or would like additional functionality let us know.

\section{PIO pins}

mat91lib provides efficient PIO abstraction routines in
\wfile{mat91lib/sam4s/pio.h}.  Each pin can be configured as follows:
%
\begin{minted}{C}
    PIO_INPUT,              /* Configure as input pin.  */
    PIO_PULLUP,             /* Configure as input pin with pullup.  */
    PIO_PULLDOWN,           /* Configure as input pin with pulldown.  */
    PIO_OUTPUT_LOW,         /* Configure as output, initially low.  */
    PIO_OUTPUT_HIGH,        /* Configure as output, initially high.  */
    PIO_PERIPH_A,           /* Configure as peripheral A.  */
    PIO_PERIPH_A_PULLUP,    /* Configure as peripheral A with pullup.  */
    PIO_PERIPH_B,           /* Configure as peripheral B.  */
    PIO_PERIPH_B_PULLUP,    /* Configure as peripheral B with pullup.  */
    PIO_PERIPH_C,           /* Configure as peripheral C.  */
    PIO_PERIPH_C_PULLUP     /* Configure as peripheral C with pullup.  */
\end{minted}

Here's an example:
%
\begin{minted}{C}
  #include "pio.h"

  // Configure PA0 as an output and set default state to low.
  pio_config_set (PIO_PA0, PIO_OUTPUT_LOW);

  // Set PA0 high.
  pio_output_high (PIO_PA0);

  // Set PA0 low.
  pio_output_low (PIO_PA0);

  // Set PA0 to value.
  pio_output_set (PIO_PA0, value);

  // Toggle PA0.
  pio_output_toggle (PIO_PA0);

  // Reonfigure PA0 as an output connected to peripheral A.
  pio_config_set (PIO_PA0, PIO_PERIPH_A);

  // Reonfigure PA0 as an input with pullup enabled.
  pio_config_set (PIO_PA0, PIO_INPUT_PULLUP);

  // Read state of PIO pin.
  result = pio_input_get (PIO_PA0);
\end{minted}

Note, you can reconfigure a PIO pin on the fly.  For example, you may
want the pin to be driven by the PWM peripheral and then at some stage
forced low.  To do this, use \code{pwm\_config\_set}.


\section{Delaying}

mat91lib provides a macro \code{DELAY_US} in \wfile{mat91lib/delay.h}
for a busy-wait delay in microseconds (this can be a floating point
value).  The CPU busy-waits for a precomputed number of clock cycles.
The argument should be a constant so the compiler can compute the
number of clock cycles.

mat91lib also provides a function \code{delay_ms} for a busy-wait
delay in milliseconds (this must be an integer).  All this function
does is call \code{DELAY_US (1000)} the required number of times.

An example program is \wfile{test-apps/delay_test1/delay_test1.c}.



\section{Miscellaneous}

\subsection{Disabling JTAG pins}
\label{disabling-jtag-pins}

By default \pin{PB4} and \pin{PB5} are configured as JTAG pins. You can turn
them into PIO pins or use them for TWI1 using:
%
\begin{minted}{C}
#include "mcu.h"

void main (void)
{
    mcu_jtag_disable ();
}
\end{minted}


\subsection{Watchdog timer}
\label{watchdog-timer}

The watchdog timer is useful for resetting the SAM4S if it hangs in a
loop.  It is disabled by default but can be enabled using:
%
\begin{minted}{C}
#include "mcu.h"

void main (void)
{
    mcu_watchdog_enable ();

    while (1)
    {
         /* Do your stuff here.  */

         mcu_watchdog_reset ();
    }
}
\end{minted}


\section{I/O model}


The mat91lib and mmculib libraries try to use the POSIX I/O model.  In
general, non-blocking I/O is used so if data is not available, the
functions return without waiting, or with a short timeout.

The read/write functions have three arguments:
%
\begin{enumerate}
\item A device handle created by the init function.
\item A pointer to a buffer.
\item The number of bytes to transfer (type \code{size_t})
\end{enumerate}
%
The return value is of type \code{ssize_t} and is either:
%
\begin{enumerate}
\item The number of bytes transferred if greater than or equal to zero.
\item -1 for error, in which case the error number is stored in \code{errno}.
\end{enumerate}

Here is an example:

\begin{minted}{C}
  char *buffer[4];

  ret = adc_read (adc, buffer, sizeof (buffer));
\end{minted}


\section{Coding style}

For the curious, the GNU C coding style is used.  I worked as a
compiler engineer for the USA company Cygnus Solutions (taken over by
Redhat) on the GNU C compiler (GCC).  The first-rule when working as a
programmer for a company is to follow the required coding style.  My
text editor (emacs) is set up to default to this style.


\section{Under the bonnet}

The building is controlled by \wfile{mat91lib/mat91lib.mk}. This is a
makefile fragment loaded by \wfile{mmculib/mmculib.mk}.
\file{wacky-reacers/src/mat91lib/mat91lib.mk} loads other makefile
fragments for each peripheral or driver required. It also
automatically generates dependency files for the gazillions of other
files that are required to make things work.
